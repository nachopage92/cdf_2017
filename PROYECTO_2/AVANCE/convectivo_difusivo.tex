\section{Desarrollo: Discretización de ecuaciones}

Ecuación de conservación/transporte de un escalar pasivo:

\begin{equation} \label{transporte}
\dfrac{\partial \rho \phi}{\partial t} + \nabla \cdot J = \vec{\nabla} P 
\end{equation}

donde $J$ representa la contribución del flujo convectivo y difusivo.

\begin{equation}
J = \rho u \phi - \Gamma \Delta \phi
\end{equation}

Para obtener la formulación en volumenes finitos se aplica el método de residuos ponderados a la ecuación, con soporte compacto y utilizando el método de Galerkin,

\begin{equation}\dfrac{\partial}{\partial t} \iiint_{\Omega_{cv}} \rho \phi d \Omega = - \oiint_{A_{cv}} \vec{F} \cdot \vec{n} d A_n + \iiint_{\Omega_{cv}} S_{\phi} d \Omega
\end{equation}  

Donde

\begin{equation}
\vec{F} = \underbrace{\vec{F}_C}_{\rho \phi \vec{u}} + \underbrace{\vec{F}_D}_{-D \nabla \phi}
\end{equation}

Se recurre al teorema de valor medio representar la integral en función de terminos centrales. Sea $\phi_J$ un valor aproximado que caracteriza al escalar dentro del volumen de control $\Omega_{cv} = \Omega_J$. La ecuación discretizada resulta

\begin{equation}
\dfrac{\partial}{\partial t} \left( \rho \phi_J \Omega_J \right) + \sum \left( F_i \, A_i \right)_J = \left( S_{\phi} \right)_J
\end{equation} 

\subsection{Discretización del flujo convectivo/difusivo en 2D}

Se escoge la discretización propuesta por Patankar \cite{patankar} para dominios en dos dimensiones ($\Delta z = 1$):

\begin{equation} \label{ecuacion_patankar}
\dfrac{\partial \phi}{\partial t} \Delta x \Delta y = a_E \phi_E + a_W \phi_W + a_N \phi_N + a_S \phi_S + a_E \phi_E + a_P \phi_P + b
\end{equation} 

Los coeficientes $a$ están dados por,

\begin{align} \label{coeficientes_patankar}
\begin{split}
a_E &= D_e A(|P_e|) + \mbox{max}(-F_e,0) \\
a_W &= D_w A(|P_w|) + \mbox{max}(F_w,0) \\
a_N &= D_n A(|P_n|) + \mbox{max}(-F_n,0) \\
a_S &= D_s A(|P_s|) + \mbox{max}(F_s,0) \\
a_P &= -a_E -a_W -a_N -a_S \\
b &= S \Delta x \Delta y
\end{split}
\end{align}

\begin{align}
\begin{split}
F_e &= (\rho \, u)_e \, \Delta y \\
F_w &= (\rho \, u)_w \, \Delta y \\
F_n &= (\rho \, v)_n \, \Delta x \\
F_s &= (\rho \, v)_s \, \Delta x
\end{split}
\end{align}

\begin{align}
\begin{split}
D_e &= \dfrac{\Gamma_e}{\Delta x \, \rho \, u_e} \Delta y = \dfrac{1}{Re_e} \Delta y \\
D_w &= \dfrac{\Gamma_w}{\Delta x \, \rho \, u_w} \Delta y = \dfrac{1}{Re_w} \Delta y \\
D_n &= \dfrac{\Gamma_n}{\Delta y \, \rho \, v_n} \Delta x = \dfrac{1}{Re_n} \Delta x \\
D_s &= \dfrac{\Gamma_s}{\Delta y \, \rho \, v_s} \Delta x = \dfrac{1}{Re_s} \Delta x
\end{split}
\end{align}

\begin{align}
\begin{split}
P_e &= \dfrac{F_e}{D_e} \\
P_w &= \dfrac{F_w}{D_w} \\
P_n &= \dfrac{F_n}{D_n} \\
P_s &= \dfrac{F_s}{D_s}
\end{split}
\end{align}

Donde $S$ representa las fuerzas externas independientes de $\phi$. Particularmente estará asociado al gradiente de presión $\vec{\nabla} P$. los términos $F_{nb}$, $D_{nb}$ y $P_{nb}$ están asociados al flujo convectivo, flujo difusivo y el número de Peclet, respectivamente. En la Tabla \ref{tabla_patankar} se exponen distintas fórmulas para calcular la función $A(|P|)$. 

\begin{table} [H]
\centering
\begin{tabular}{|l|l|} \hline
Esquema & Fórmula para $A(|P|)$ \\ \hline \hline
CDS	&	$1-0.5|p|$	\\
UDS	&	$1$	\\
Hybrid	&	$\mbox{max}(0,1-0.5|P|)$	\\
Power law	&	$\mbox{max}(0,(1-0.1|P|)^5)$	\\
Exponential	&	$|P|/\left[ \mbox{exp}(|P|) -1 \right]$ \\ \hline
\end{tabular}
\caption{Funciones de $A(|P|)$ para distintos esquemas de discretización \cite{patankar}. ($P$: número de Peclet)} \label{tabla_patankar}
\end{table}

Este conjunto de ecuaciones discretiza el campo $\phi$ mediante la aplicación de las ecuaciones de conservación de la masa y la ecuación de conservación de la cantidad lineal. Sea $\vec{v} = \{u,v\}^T$ el campo de velocidad en el dominio de control entonces se plantean los sistemas de ecuaciones donde $\phi=u$ y $\phi=v$, donde $u$ y $v$ son escalares asociados a cada malla desplazada (\textit{staggered grid}). Los coeficientes de la ecuación (\ref{ecuacion_patankar}) se calculan de acuerdo a la malla desplaza asociada a la variable $\phi$ a evaluar. Si $\phi = u$, entonces:

\begin{equation}
\dfrac{\partial u}{\partial t} \Delta x \Delta y = a_E^u u_E + a_W^u u_W + a_N^u u_N + a_S^u u_S + a_P^u u_P + b^u
\end{equation} 

Análogo para $\phi = v$

\subsubsection{Término convectivo} \label{convectivo}

Agrupando los términos convectivos de la ecuación \ref{ecuacion_patankar} se obtiene,

\begin{equation}
\begin{split}
H(u) = \underbrace{\mbox{max}(F_w^u,0) u_W \Delta y - (\mbox{max}((F_w^u,0)) + \mbox{max}(-F_e^u,0)) u_P \Delta y + \mbox{max}(-F_e^u,0) u_E \Delta y}_{\partial u^2 / \partial x} + \\ \underbrace{\mbox{max}(F_s^u,0) u_S \Delta x - (\mbox{max}((F_s^u,0)) + \mbox{max}(-F_n^u,0)) u_P \Delta x + \mbox{max}(-F_n^u,0) u_N \Delta x}_{\partial uv / \partial y}
\end{split}
\end{equation}

\begin{equation}
\begin{split}
H(v) =  \underbrace{\mbox{max}(F_w^v,0) v_W \Delta y - (\mbox{max}((F_w^v,0)) + \mbox{max}(-F_e^v,0)) v_P \Delta y + \mbox{max}(-F_e^v,0) v_E \Delta y}_{\partial uv / \partial x} + \\ \underbrace{\mbox{max}(F_s^v,0) v_S \Delta x - (\mbox{max}((F_s^v,0)) + \mbox{max}(-F_n^v,0)) v_P \Delta y + \mbox{max}(-F_n^v,0) v_N \Delta x}_{\partial v^2 / \partial y} 
\end{split}
\end{equation}

Por simplicidad,

\begin{equation}
H(u) = h_W^u u_W + h_E^u u_E + h_S^u u_S + h_N^u u_N + h^u_P u_P
\end{equation}

\begin{equation}
H(v) = h_W^v v_W + h_E^v v_E + h_S^v v_S + h_N^v v_N + h_P^v v_P
\end{equation}

\subsubsection{Término difusivo} \label{difusivo}

Agrupando los términos difusivos de la ecuación \ref{ecuacion_patankar} se obtiene,

\begin{equation}
\begin{split}
G(u) = \dfrac{1}{Re_w^u} A(|P_w^u|) u_W \Delta y + \dfrac{1}{Re_e^u} A(|P_e^u|) u_E \Delta y \\ + \dfrac{1}{Re_s^u} A(|P_s^u|) u_S \Delta x + \dfrac{1}{Re_n^u} A(|P_n^u|) u_N \Delta x \\ - \left[ \left( \dfrac{1}{Re_n^u} A(|P_n^u|) + \dfrac{1}{Re_s^u} A(|P_s^u|) \right) \Delta x \right. \\ + \left. \left( \dfrac{1}{Re_w^u} A(|P_w^u|) + \dfrac{1}{Re_e^u} A(|P_e^u|) \right) \Delta y \right] u_P
\end{split}
\end{equation}

\begin{equation}
\begin{split}
G(v) = \dfrac{1}{Re_w^v} A(|P_w^v|) v_W \Delta y + \dfrac{1}{Re_e^v} A(|P_e^v|) v_E \Delta y \\ + \dfrac{1}{Re_s^v} A(|P_s^v|) v_S \Delta x + \dfrac{1}{Re_n^v} A(|P_n^v|) v_N \Delta x \\ - \left[ \left( \dfrac{1}{Re_n^v} A(|P_n^v|) + \dfrac{1}{Re_s^v} A(|P_s^v|) \right) \Delta x \right. \\ + \left. \left( \dfrac{1}{Re_w^v} A(|P_w^v|) + \dfrac{1}{Re_e^v} A(|P_e^v|) \right) \Delta y \right] v_P
\end{split}
\end{equation}

Por simplicidad,

\begin{equation}
G(u) = g_W^u u_W + g_E^u u_E + g_S^u u_S + g_N^u u_N + g_P^u u_P
\end{equation}

\begin{equation}
G(v) = g_W^v v_W + g_E^v v_E + g_S^v v_S + g_N^v v_N + g_P^v v_P
\end{equation}

