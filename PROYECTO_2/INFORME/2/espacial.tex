%---------------------------------------------------

\subsection{Discretización espacial}

Ecuación de conservación/transporte de un escalar pasivo $\phi$:

\begin{equation} \label{transporte}
\dfrac{\partial \phi}{\partial t} + \nabla \cdot J = S_{\phi}
\end{equation}

donde $S_{\phi}$ representa el término fuente asociado a $\phi$ y $J$ a la contribución del flujo convectivo y difusivo:

\begin{equation}
J = \vect{v} \phi - \dfrac{1}{Re} \vec{\nabla} \phi
\end{equation}

Para obtener la formulación en volumenes finitos se aplica el método de residuos ponderados a la ecuación, con soporte compacto y utilizando el método de Galerkin [CITAR LIBRO]. Se recurre al teorema de valor medio representar la integral en función de terminos centrales. Sea $\phi_J$ un valor aproximado que caracteriza al escalar dentro del volumen de control $\Omega_{cv} = \Omega_J$. La ecuación discretizada resulta [CITAR LIBRO]:

\begin{equation}
\dfrac{\partial}{\partial t} \left( \phi_J \Omega_J \right) + \sum \left( F_i \, A_i \right)_J = \left( S_{\phi} \right)_J
\end{equation} 

Donde $F_i$ corresponde la contribución del flujo convectivo y difusivo en la cara $A_i$.
\begin{equation}
F = \underbrace{H(\phi)_C}_{\vect{v} \phi} + \underbrace{G(\phi)_D}_{-\vec{\nabla} \phi / Re}
\end{equation}

%---------------------------------------------------

\subsubsection{Término convectivo}

La discretización del término convectivo viene dado por
\begin{equation}
\iiint\limits_{\Omega} H(\phi) \, dV = \iiint\limits_{\Omega} \nabla \cdot \left( \vect{v} \phi \right) \, dV = \iint\limits_{\partial \Omega} \vect{v} \phi \cdot \vect{n} \, dA
\end{equation}

Aplicando el método de volumenes finitos resulta la siguiente expresión:

\begin{equation}
\iint\limits_{\partial \Omega} \vect{v} \phi \cdot \vect{n} \, dA \approx \left[ (\phi u)_e -(\phi u)_w \right] \Delta y + \left[ (\phi v)_n -(\phi v)_s \right] \Delta x
\end{equation}

Donde,

\begin{equation}
\begin{split}
&(u)_e = \frac{1}{2}(u_E+u_P) \\
&(u)_w = \frac{1}{2}(u_P+u_W) \\
&(v)_n = \frac{1}{2}(v_N+v_P) \\
&(v)_s = \frac{1}{2}(v_P+v_S)
\end{split}
\end{equation}

\begin{equation}
\begin{split}
&(\phi)_e = \frac{1}{2}(\phi_E+\phi_P) \\
&(\phi)_w = \frac{1}{2}(\phi_P+\phi_W) \\
&(\phi)_n = \frac{1}{2}(\phi_N+\phi_P) \\
&(\phi)_s = \frac{1}{2}(\phi_P+\phi_S)
\end{split}
\end{equation}

Luego,

\begin{equation}
\begin{split}
\iiint\limits_{\Omega} H(\phi) \, dV \approx \left[ \dfrac{\phi_E+\phi_P}{2} \dfrac{u_E+u_P}{2} - \dfrac{\phi_P+\phi_W}{2} \dfrac{u_P+u_W}{2} \right] \Delta y \\
+ \left[ \dfrac{\phi_N+\phi_P}{2} \dfrac{v_N+v_P}{2} - \dfrac{\phi_P+\phi_S}{2} \dfrac{v_P+v_S}{2} \right] \Delta x 
\end{split}
\end{equation}

Particularmente si se considera $\vect{H}(\vect{v}) = (h(u),h(v))$ se tiene que 

\begin{equation}
\begin{split}
h(u) \approx \underbrace{ \dfrac{1}{\Delta x} \left[ \dfrac{u_E+u_P}{2} \dfrac{u_E+u_P}{2} - \dfrac{u_P+u_W}{2} \dfrac{u_P+u_W}{2} \right]}_{\partial u^2 / \partial x} \, \Delta x \, \Delta y  \\
+ \underbrace{ \dfrac{1}{\Delta y} \left[ \dfrac{u_N+u_P}{2} \dfrac{v_N+v_P}{2} - \dfrac{u_P+u_S}{2} \dfrac{v_P+v_S}{2} \right] }_{\partial uv / \partial y} \, \Delta x \, \Delta y
\end{split}
\end{equation}

\begin{equation}
\begin{split}
h(v) \approx \underbrace{ \dfrac{1}{\Delta x} \left[ \dfrac{v_E+v_P}{2} \dfrac{u_E+u_P}{2} - \dfrac{v_P+v_W}{2} \dfrac{u_P+u_W}{2} \right]}_{\partial uv / \partial x} \, \Delta x \, \Delta y \\
+ \underbrace{ \dfrac{1}{\Delta y} \left[ \dfrac{v_N+v_P}{2} \dfrac{v_N+v_P}{2} - \dfrac{\phi_P+\phi_S}{2} \dfrac{v_P+v_S}{2} \right] }_{\partial v^2 / \partial y} \, \Delta x \, \Delta y
\end{split}
\end{equation}

	
%-------------------------------------------------

\subsubsection{Término difusivo}

Análogo a la discretización anterior, se aproxima el término difusivo utilizando el método de volúmenes finitos.

\begin{equation}
\iiint\limits_{\Omega} G(\phi) \, dV = \iiint\limits_{\Omega} \nabla \cdot \vec{\nabla} \phi \, dV = \iint\limits_{\partial \Omega} \vec{\nabla} \phi \cdot \vect{n} \, dA 
\end{equation}

Donde,

\begin{equation}
\iint\limits_{\partial \Omega} \vec{\nabla} \phi \cdot \vect{n} \, dA \approx \dfrac{1}{Re} \left[ (\vec{\nabla} \phi)_e - (\vec{\nabla} \phi)_w \right] \Delta y + \dfrac{1}{Re} \left[ (\vec{\nabla} \phi)_n - (\vec{\nabla} \phi)_s \right] \Delta x
\end{equation}

\begin{equation}
\begin{split}
&(\vec{\nabla} \phi)_e = \frac{1}{\Delta x}(\phi_E-\phi_P) \\
&(\vec{\nabla} \phi)_w = \frac{1}{\Delta x}(\phi_P-\phi_W) \\
&(\vec{\nabla} \phi)_n = \frac{1}{\Delta y}(\phi_N-\phi_P) \\
&(\vec{\nabla} \phi)_s = \frac{1}{\Delta y}(\phi_P-\phi_S)
\end{split}
\end{equation}

Reordenando los términos se obtiene:

\begin{equation}
\iiint\limits_{\Omega} G(\phi) \, dV \approx \dfrac{1}{Re} \dfrac{\Delta y}{\Delta x} \left[ \phi_E - 2\phi_P + \phi_w \right] + \dfrac{1}{Re} \dfrac{\Delta x}{\Delta y} \left[ \phi_N - 2\phi_P + \phi_S \right]
\end{equation}
