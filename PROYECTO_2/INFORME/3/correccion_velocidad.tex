\subsection{Correción del campo de velocidad}

La corrección de la presión es un proceso iterativo donde se vuelve a calcular el predictor de la velocidad $\vect{v}^*$. Como criterio de detención se reemplaza la ecuación (\ref{poisson}) en la ecuación de conservación de masa

\begin{equation}
\nabla \cdot \vect{v}^{n+1} = \nabla \cdot \vect{v}^* - \underbrace{ \dfrac{2 \, \Delta t}{3} \Delta \phi }_{\mbox{fuente de masa}} = 0
\end{equation}

Cuando el término de fuente de masa es igual a cero entonces se puede decir que las variables iteradas convergen. Luego se corrige el campo de velocidad $\vec{v}^{n+1} = \vec{v} - \Delta t \nabla \phi$. Alcanzada la convergencia se pasa al siguiente pasa de tiempo hasta terminar la iteración al tiempo $T$.