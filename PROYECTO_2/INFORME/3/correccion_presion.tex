\subsection{Resolución de la ecuación de Poisson sobre la presión}

\begin{equation}
\Delta \phi = \dfrac{3}{2 \Delta t} \nabla \cdot \vec{v}^*
\end{equation}

\begin{equation}
\phi = \left( P^{n+1} - P^n \right) + \dfrac{1}{Re} \nabla \cdot \vec{v}^* 
\end{equation}

Nuevamente se aplica el método de residuos ponderados con formulación para volúmenes finitos.

\begin{equation}
\iiint_{\Omega} \Psi_i  \left[ \Delta \phi - \dfrac{3}{2 \Delta t} \nabla \cdot v_i^* \right] \, dV = 0
\end{equation}

Luego,

\begin{equation}
\iiint_{\Omega_J} \Delta \phi \, dV = \iiint_{\Omega_J} \dfrac{3}{2 \Delta t} \nabla \cdot v_i^* \, dV
\end{equation}

Recurriendo al teorema de Green-Ostrogradsky se rescribe la ecuación:

\begin{equation}
\iint_{\partial \Omega_J} \vec{\nabla} \phi \vec{n} \, dA = \dfrac{3}{2 \Delta t} \iint_{\partial \Omega_J} \vec{v}^* \cdot \vec{n} \, dA
\end{equation}

\paragraph{Discretización} La función auxiliar $\phi$ se discretiza en la malla desplazada asociada a la presión. Luego, los valores de los flujos de $\phi$ se aproximan en los bordes de cada volumen finito $\Omega_J$, mientras que los valores de las velocidad en $\partial \Omega_J$ son conocidos, sin necesidad de aproximarlos.

\begin{equation}
\begin{split}
\left[ \dfrac{\partial \phi}{\partial x} \Big|_e - \dfrac{\partial \phi}{\partial x} \Big|_w \right] \Delta y + \left[ \dfrac{\partial \phi}{\partial y} \Big|_n - \dfrac{\partial \phi}{\partial y} \Big|_s \right] \Delta x = \\ \dfrac{3}{2 \Delta t} \left( u^*_e - u_w^*  \right) + \dfrac{3}{2 \Delta t} \left( v^*_n - v_s^*  \right)
\end{split}
\end{equation}

Se discretizan las derivadas de $\phi$ implementando un esquemas CDS, obteniendose:

\begin{equation} \label{poisson_discreto}
\begin{split}
\left[ \dfrac{\phi_E-2\phi_P+\phi_W}{\Delta x} \right] \Delta y - \left[ \dfrac{\phi_N-2\phi_P+\phi_S}{\Delta y} \right] \Delta x = \\ \dfrac{3}{2 \Delta t} \left( u_e^* - u_w^* + v_n^* - v_s^* \right)
\end{split}  
\end{equation}

Agrupando los términos $\phi_{nb}$ resulta un sistema de ecuaciones lineales que en forma matricial se representa por una matriz pentadiagonal. Para resolverlo se utiliza el algoritmo TDMA (\textit{Tri-Diagonal MAtrix Algorithm}). Para ellos se define el término $B$ como,

\begin{equation}
B =  \dfrac{3}{2 \Delta t} \left( u_e^* - u_w^* + v_n^* - v_s^* \right)
\end{equation}

Se calcula un predictor (I):

\begin{equation}
\begin{split}
\left( \dfrac{\Delta y}{\Delta x} \right) \phi_E^I - \left( 2 \dfrac{\Delta y}{\Delta x} + 2 \dfrac{\Delta x}{\Delta y} \right) \phi_P^I + \left( \dfrac{\Delta y}{\Delta x} \right) \phi_W^I = \\ \left( \dfrac{\Delta x}{\Delta y} \right) \phi_N + \left( \dfrac{\Delta x}{\Delta y} \right) \phi_S + B
\end{split}
\end{equation}

Se calcula una corrección (II):

\begin{equation}
\begin{split}
\left( \dfrac{\Delta x}{\Delta y} \right) \phi_N^{II} - \left( 2 \dfrac{\Delta y}{\Delta x} + 2 \dfrac{\Delta x}{\Delta y} \right) \phi_P^{II} + \left( \dfrac{\Delta x}{\Delta y} \right) \phi_S^{II} = \\ \left( \dfrac{\Delta y}{\Delta x} \right) \phi_E^I + \left( \dfrac{\Delta y}{\Delta x} \right) \phi_W^I + B
\end{split}
\end{equation}

Un criterio de detención que suele utilizarse en establece lo siguiente: En cada punto de la malla se calcula un residuo $R$ como,

\begin{equation}
R = \sum a_{nb} \phi_{nb} + B - a_P \phi_P 
\end{equation}

La solución converge en la medida que $R$ tiende a cero. Más detalles del algoritmo se explican en \cite{patankar} y \cite{versteeg}. 
