\subsection{Corrección de la presión}

Una vez calculada la predicción de velocidad se corrige el campo de presión $P$ establecido \textit{a priori}. Como $G(\vect{v}) = \frac{1}{Re} \Delta \vect{v}$ entonces se escribe la ecuación de conservación de momento para la predicción de $\vect{v}^*$ y para el instante de tiempo $t=n+1$:

\begin{equation}
\dfrac{3 \vect{v}^* -4 \vect{v}^n + \vect{v}^{n-1} }{2 \Delta t} + H(\vect{v}^{n+1}) = - \vec{\nabla} P^n + \dfrac{1}{Re} \nabla \vect{v}^*
\end{equation}

\begin{equation}
\dfrac{3 \vect{v}^{n+1} -4 \vect{v}^n + \vect{v}^{n-1} }{2 \Delta t} + H(\vect{v}^{n+1}) = - \vec{\nabla} P^{n+1} + \dfrac{1}{Re} \nabla \vect{v}^{n+1}
\end{equation}

Restando ambas ecuaciones se obtiene

\begin{equation} \label{def_phi}
\dfrac{3}{2 \, \Delta t} \left( \vect{v}^{n+1} - \vect{v}^* \right) = - \underbrace{ \left[ \vec{\nabla} P^{n+1} - \vec{\nabla} P^n - \dfrac{1}{Re} (\Delta \vect{v}^{n+1} - \Delta \vect{v}^* ) \right] }_{\nabla \phi}
\end{equation}

Al aplicar el operador divergencia en ambos lados de la igualdad y considerando que $\nabla \cdot \vect{v}^{n+1} = 0$ resulta entonces:

\begin{equation} \label{poisson}
\Delta \phi = \dfrac{3}{2 \Delta t} \nabla \cdot \vec{v}^*
\end{equation}

Tomando el lado derecho de la ecuación (\ref{def_phi}) y desarrollando el operador laplaciano de $\vect{v}$ se observa que:

\begin{equation}
\nabla \phi = \vec{\nabla} P^{n+1} - \vec{\nabla} P^n - \dfrac{1}{Re} ( \nabla ( \nabla \cdot \vect{v}^{n+1} ) - \nabla \times (\nabla \times \vect{v}^{*}) - \nabla ( \nabla \cdot \vect{v}^{n+1} ) + \nabla \times (\nabla \times \vect{v}^{*})  )
\end{equation}

Si imponemos que $\nabla \cdot \vect{v}^{n+1} = 0$ y $\nabla \times (\nabla \times \vect{v}^{*}) = \nabla \times (\nabla \times \vect{v}^{n+1})$ entonces la expresión se reduce a:

\begin{equation} \label{phi_explicito}
\phi = \left( P^{n+1} - P^n \right) + \dfrac{1}{Re} \nabla \cdot \vec{v}^* 
\end{equation}

Aplicando el método de volúmenes finitos a la ecuación (\ref{poisson}) se tiene entonces,

\begin{equation}
\left( \dfrac{\phi_E-2\phi_P+\phi_W}{(\Delta x)^2} + \dfrac{\phi_N-2\phi_P+\phi_S}{(\Delta y)^2} \right) \, \Delta x \Delta y = \dfrac{3}{2 \, \Delta t} \left[ ( u^*_e - u^*_w ) \Delta y + ( v^*_n - v^*_s ) \Delta x \right]
\end{equation} 

Obviando los términos vecinos del punto central $\phi_P$ entonces la ecuación se puede simplificar,

\begin{equation}
\phi_P = - \frac{\dfrac{3}{2 \, \Delta t} \left[ ( u^*_e - u^*_w ) \Delta y + ( v^*_n - v^*_s ) \Delta x \right]}{ \frac{1}{2} ( \frac{\Delta y}{\Delta x} + \frac{\Delta x}{\Delta y} ) }
\end{equation}

De esta manera se evita imponer condiciones de contorno para $\phi$, ya que estas se desprenden implícitamente de las condiciones de borde y resolución de la variable $\vect{v}$. Finalmente la corrección de la presión se realiza despejando $P^{n+1}$ de la ecuación (\ref{phi_explicito}). 

