\subsection{Predicción del campo de velocidad no solenoidal $\vect{v}^*$}

La resolución de (\ref{ecuación_gobernante}) requiere del cálculo de un predictor de la velocidad $\vect{v}$ dado por:

\begin{equation}
\dfrac{3 \vect{v}^{*} -4\vect{v}^n + \vect{v}^{n-1}}{2 \Delta t} + H(\vect{v}^{n+1}) = -\vec{\nabla} P + G(\vect{v}^{*})
\end{equation}

Sea $\vect{\delta V} = \vect{v}^* - \vect{v}^n $ y mediante una manipulación algebráica de la ecuación (\ref{ecuación_gobernante}) se obtiene una ecuación equivalente.

\begin{equation} \label{ecuación_gobernante_modificada}
\left( I - \dfrac{2 \Delta t}{3 \, Re} \Delta \right) \vect{\delta V} = \dfrac{\vect{v}^n-\vect{v}^{n-1}}{3} - \dfrac{2 \Delta t}{3} \left( 2 H(\vect{v}^n) - H(\vect{v}^{n-1}) + \vec{\nabla} P - G(\vect{v}^n) \right)
\end{equation} 

Aplicando el método ADI (\textit{Alternating Direction Implicit}) para descomponer al operador de Helmholtz ($I - \frac{2 \Delta t}{3 \, Re} \Delta $)

\begin{equation}
\left( I - \dfrac{2 \Delta t}{3 \, Re} \Delta \right) \approx \left( I - \dfrac{2 \Delta t}{3 \, Re} \dfrac{\partial^2}{\partial y^2} \right) \left( I - \dfrac{2 \Delta t}{3 \, Re} \dfrac{\partial^2}{\partial x^2} \right)
\end{equation}

Luego, resolver (\ref{ecuación_gobernante_modificada}) es equivalente de manera aproximada a resolver

\begin{equation}
\left( I - \dfrac{2 \Delta t}{3 \, Re} \dfrac{\partial^2}{\partial y^2} \right)  \left( I - \dfrac{2 \Delta t}{3 \, Re} \dfrac{\partial^2}{\partial x^2} \right) \delta V =  \dfrac{\vec{v}^n-\vec{v}^{n-1}}{3} - \dfrac{2 \Delta t}{3} \left( H(\vec{v}^{n+1}) + \vec{\nabla} P) \right)
\end{equation}

O bien,

\begin{align}
\left( I - \dfrac{2 \Delta t}{3 \, Re} \dfrac{\partial^2}{\partial x^2} \right) \delta \overline{V} &= RHS^n \label{primer_paso} \\ 
\left( I - \dfrac{2 \Delta t}{3 \, Re} \dfrac{\partial^2}{\partial y^2} \right) \delta V &= \overline{V} 
\end{align}

\subsubsection{Primer paso en la dirección $x$ ($\delta \overline{V}$)} 

Explicitando los términos de $RHS^n$ de la ecuación (\ref{primer_paso})

\begin{equation}
\left( I - \dfrac{2 \Delta t}{3 \, Re} \dfrac{\partial^2}{\partial x^2} \right) \delta \overline{V} = \dfrac{\vec{v}^n-\vec{v}^{n-1}}{3} - \dfrac{2 \Delta t}{3} \left( H(\vec{v}^{n+1}) + \vec{\nabla} P) \right) = RHS^n
\end{equation}

Recurriendo al método de residuos ponderados,

\begin{equation} 
\iiint_{\Omega} \Psi_i \left[ \left( I - \dfrac{2 \Delta t}{3 \, Re}  \dfrac{\partial^2}{\partial x^2} \right) \delta \overline{V}_i - RHS^n \right] \, dV = 0
\end{equation}

Y aplicando la formulación para volúmenes finitos, desarrrollando el primer término de la izquierda

\begin{equation}
\iiint_{\Omega_{cv}} \left( I - \dfrac{2 \Delta t}{3 \, Re}  \dfrac{\partial^2}{\partial x^2} \right) \delta \overline{V} \, dV =  \underbrace{\iiint_{\Omega_{cv}} \delta \overline{V} \, dV}_{(*)}  - \underbrace{\dfrac{2 \Delta t}{3 \, Re} \iiint_{\Omega_{cv}}  \dfrac{\partial^2 (\delta \overline{V}) }{\partial x^2} \, dV}_{(**)}
\end{equation}

Resolviendo $(*)$

\begin{equation}
\iiint_{\Omega_{cv}} \delta \overline{V} \, dV \approx  (\delta \overline{V})_P \, \Delta x \Delta y
\end{equation}

Resolviendo $(**)$

\begin{equation}
\dfrac{2 \Delta t}{3 \, Re} \iiint_{\Omega_{cv}}  \dfrac{\partial^2 (\delta \overline{V}) }{\partial x^2} \, dV \approx \dfrac{2 \, \Delta t}{3 \, Re} \dfrac{\Delta y}{\Delta x} \left( \delta \overline{V}_E -2 \delta \overline{V}_P + \delta \overline{V}_W \right)
\end{equation}

Reemplazando $(**)$ y $(***)$ en la (\ref{primer_paso}) se obtiene:

\begin{equation}
(\delta \overline{V})_P \, \Delta x \Delta y - \dfrac{2 \, \Delta t}{3 \, Re} \dfrac{\Delta y}{\Delta x} \left( \delta \overline{V}_E -2 \delta \overline{V}_P + \delta \overline{V}_W \right) = \iiint_{\Omega_{cv}} RHS^n \, dV
\end{equation}

El término $RHS^n$ se reemplaza por la ecuación la discretización planteada en la Sección \ref{seccion2} y por el campo de presión definido arbitrariamente a \textit{priori}. La ecuación anterior se traduce en resolver un sistemas de ecuaciones tridiagonal dado por,

\begin{equation} \label{hola}
\begin{split}
\left( - \dfrac{2 \, \Delta t}{3 \, Re} \dfrac{\Delta y}{\Delta x}  \right) \delta \overline{V}_E + \left( \Delta x \, \Delta y + \dfrac{4 \, \Delta t}{3 \, Re} \dfrac{\Delta y}{\Delta x}  \right) \delta \overline{V}_P + \left( - \dfrac{2 \, \Delta t}{3 \, Re} \dfrac{\Delta y}{\Delta x}  \right) \delta \overline{V}_W \\
 = \iiint_{\Omega_{cv}} RHS^n \, dV
\end{split}
\end{equation}

\subsubsection{Segundo paso en la dirección $y$ ($\delta V$)}

Se procede similar al paso anterior, con la diferencia de que no se recurre al método de volúmenes finitos, ya que se aplica sólo a un factor de la descomposición.  

\begin{equation}
(\delta \overline{V})_P - \dfrac{2 \, \Delta t}{3 \, Re} \dfrac{1}{(\Delta y)^2} \left( \delta V_E -2 \delta V_P + \delta V_W \right) = \delta \overline{V}_P
\end{equation}

Reeordenando los términos para obtener un sistema de ecuaciones tridiagonal

\begin{equation}
\begin{split}
\left( - \dfrac{2 \, \Delta t}{3 \, Re} \dfrac{1}{(\Delta y)^2}  \right) \delta V_E + \left( 1 + \dfrac{4 \, \Delta t}{3 \, Re} \dfrac{1}{(\Delta y)^2}  \right) \delta V_P + \left( - \dfrac{2 \, \Delta t}{3 \, Re} \dfrac{1}{(\Delta y)^2}  \right) \delta V_W \\
 = \delta \overline{V}_P
\end{split}
\end{equation}

Los pasos anteriores se realizan tanto para $u$ como para $v$ ya que $\vect{v}=\left\{ u,v \right\}^T$ y $\delta \vect{V} = \vect{v}^* - \vect{v}^n $. Cada paso de tiempo implica resolver 4 sistemas tridiagonales: 2 sistemas para la componente horizontal y 2 sistemas para la componente vertical de la velocidad. Finalmente se obtiene el predictor de la velocidad:

\begin{equation}
\vect{v}^* = \Big| \begin{matrix} u^* = u^n + \delta V_x \\ v^* = v^n + \delta V_y
\end{matrix}
\end{equation}

\paragraph{Observación} En la ecuación (\ref{hola}) se deben considerar los esfuerzos cortantes debido las fuerzas externas. En caso de haber una paredse debe considerar la condición de no deslizamiento. En este trabajo se implementó una aproximación lineal de la fuerza utilizando el procedimiento explicado por Versteeg \cite{versteeg}   