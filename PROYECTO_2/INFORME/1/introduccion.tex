\section{Resumen}

La presente entrega tiene como finalidad exponer un avance en el desarrollo del Proyecto 2: \textit{Flujo monofásico newtoniano en un confluencia} de la Asignatura Dinámica de Fluidos Computacional. \\

Se estudia los esquemas de discretización flujos convectivos y difusivos, además  del esquema de integración temporal a utilizar en la implementación computacional, describiendose de manera general los pasos a seguir en la resolución de la ecuación de Navier-Stokes aplicado a un fluido newtoniano incompresible utilizando el método de volúmenes finitos, sin profundizar en los conceptos físicos propios del problema ni en los detalles de la programación.\\

Se emplea la discretización de los flujos de convección y difusión propuesta por Patankar \cite{patankar}. El dominio físico se discretiza utilizando mallas desplazadas (\textit{staggered grid}). Para la integración temporal se emplea un esquema de Euler de orden 2 o esquema de Gear. Esta discretización permite integrar un intervalo de tiempo mediante el método de paso fraccionados o método ADI.\\

Se sigue un procedimiento de resolución que consiste en:
\begin{enumerate}
\item Cálculo de un predictor del campo de velocidad $\vec{v}^*$.
\item Resolución de la ecuación de Poisson sobre el campo de presión.
\item Corrección de campo de velocidad en el paso de tiempo $t_{n+1}$.
\end{enumerate}

Además se explica de manera acotada el procedimiento que se llevara a cabo en la entraga final, basandose en el desarrollo metodológico expuesto. 

