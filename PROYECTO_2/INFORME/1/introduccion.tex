\section{Introducción}

Se estudia el perfil de velocidad de un flujo dentro de una tubería de sección circular. Se emplea el método de volúmenes finitos para la resolución numérica y se contrastan los resultados con los resultados analíticos. \\

En la Sección 2 se estudian los esquemas de discretización flujos convectivos y difusivos, además del esquema de integración temporal a utilizar en la implementación computacional, describiendose de manera general los pasos a seguir en la resolución de la ecuación de Navier-Stokes aplicado a un fluido newtoniano incompresible (Sección 3).\\

Para la integración temporal se emplea un esquema de Euler de orden 2 o esquema de Gear. Esta discretización permite integrar un intervalo de tiempo mediante el método de paso fraccionados o método ADI. En la Sección 4 se nombran algunas consideración al momento de discretizar el dominio físico, para lo cual se utilizan mallas desplazadas (\textit{staggered grid}).  Como se busca obtener un flujo completamente desarrollado se plantea condiciones de borde adecuadas para obtener estas características: Se emplean condición de presión conocida en la entrada y salida del tubo; y condición de periodicidad.\\

Los resultados númericos y teóricos se analizan en la Sección 5. Finalmente se realizan algunas observaciones en la Sección 6.

