\section{Discretización temporal}

Se plantea un esquema BFD2 (\textit{Backward Differentiation Formula 2nd Order}), tambien llamado esquema de Gear. La derivada en (\ref{ecuacion_patankar}) se aproxima mediante,

\begin{equation}
\dfrac{\partial \vec{v}}{\partial t} ^ {n+1} = \dfrac{3 \vec{v}^{n+1} -4\vec{v}^n + \vec{v}^{n-1}}{2 \Delta t} + \Theta(\Delta t^2) 
\end{equation}

El término convectivo en $t_{n+1}$ se obtiene por extrapolación:

\begin{equation}
H(\vec{v}^{n+1}) = 2 H(\vec{v}^n) - H(\vec{v}^{n-1})
\end{equation}

La ecuación a resolver mediante el método de volumenes finitos es:

\begin{equation} \label{ecuación_gobernante}
\dfrac{3 \vec{v}^{n+1} -4\vec{v}^n + \vec{v}^{n-1}}{2 \Delta t} + H(\vec{v}^{n+1}) = -\vec{\nabla} P + G(\vec{v}^{n+1})
\end{equation}

Los pasos a seguir son los siguientes
\begin{enumerate}
\item Predicción del campo de velocidad no solenoidal $\vec{v}^*$
\item Resolución de la ecuación de Poisson sobre la presión
\item Corrección del campo de velocidad
\end{enumerate}

\subsection{Predicción del campo de velocidad no solenoidal $\vec{v}^*$}

\begin{equation}
\dfrac{3 \vec{v}^{*} -4\vec{v}^n + \vec{v}^{n-1}}{2 \Delta t} + H(\vec{v}^{n+1}) = -\vec{\nabla} P + G(\vec{v}^{*})
\end{equation}

Sea $\delta V = \vec{v}^* - \vec{v}^n $ y mediante una manipulación algebráica de la ecuación (\ref{ecuación_gobernante}) se obtiene una ecuación equivalente.

\begin{equation} \label{ecuación_gobernante_modificada}
\left( I - \dfrac{2 \Delta t}{3 \, Re} \Delta \right) \delta V = \dfrac{\vec{v}^n-\vec{v}^{n-1}}{3} - \dfrac{2 \Delta t}{3} \left( H(\vec{v}^{n+1}) + \vec{\nabla} P) \right)
\end{equation} 

Aplicando el método ADI (\textit{Alternating Direction Implicit}) para descomponer al operador de Helmholtz ($I - \frac{2 \Delta t}{3 \, Re} \Delta $)

\begin{equation}
\left( I - \dfrac{2 \Delta t}{3 \, Re} \Delta \right) \approx \left( I - \dfrac{2 \Delta t}{3 \, Re} \dfrac{\partial^2}{\partial y^2} \right) \left( I - \dfrac{2 \Delta t}{3 \, Re} \dfrac{\partial^2}{\partial x^2} \right)
\end{equation}

Luego, resolver (\ref{ecuación_gobernante_modificada}) es equivalente de manera aproximada a resolver

\begin{equation}
\left( I - \dfrac{2 \Delta t}{3 \, Re} \dfrac{\partial^2}{\partial y^2} \right) \underbrace{ \left( I - \dfrac{2 \Delta t}{3 \, Re} \dfrac{\partial^2}{\partial x^2} \right) \delta V }_{\Delta \delta \overline{V}} = \underbrace{ \dfrac{\vec{v}^n-\vec{v}^{n-1}}{3} - \dfrac{2 \Delta t}{3} \left( H(\vec{v}^{n+1}) + \vec{\nabla} P) \right) }_{RHS^n} 
\end{equation}

O bien

\begin{align}
\begin{split}
\left( I - \dfrac{2 \Delta t}{3 \, Re} \dfrac{\partial^2}{\partial x^2} \right) \delta \overline{V} &= RHS^n \\ 
\left( I - \dfrac{2 \Delta t}{3 \, Re} \dfrac{\partial^2}{\partial y^2} \right) \delta V &= \overline{V} \\  
\end{split}
\end{align}

\subsubsection{1er paso en la dirección $x$ ($\delta \overline{V}$)} 

\begin{equation}
\left( I - \dfrac{2 \Delta t}{3 \, Re} \dfrac{\partial^2}{\partial x^2} \right) \delta \overline{V} = \dfrac{\vec{v}^n-\vec{v}^{n-1}}{3} - \dfrac{2 \Delta t}{3} \left( H(\vec{v}^{n+1}) + \vec{\nabla} P) \right) = RHS^n
\end{equation}

Se recurre al método de residuos ponderados,

\begin{equation} \label{primer_paso}
\iiint_{\Omega} \Psi_i \left[ \left( I - \dfrac{2 \Delta t}{3 \, Re}  \dfrac{\partial^2}{\partial x^2} \right) \delta \overline{V}_i - RHS^n \right] \, dV = 0
\end{equation}

Aplicando la formulación para volúmenes finitos y desarrrollando el primer término de la izquierda

\begin{equation}
\iiint_{\Omega_{cv}} \left( I - \dfrac{2 \Delta t}{3 \, Re}  \dfrac{\partial^2}{\partial x^2} \right) \delta \overline{V} \, dV =  \underbrace{\iiint_{\Omega_{cv}} \delta \overline{V} \, dV}_{(**)}  - \underbrace{\dfrac{2 \Delta t}{3 \, Re} \iiint_{\Omega_{cv}}  \dfrac{\partial^2 (\delta \overline{V}) }{\partial x^2} \, dV}_{(***)}
\end{equation}

Resolviendo $(**)$

\begin{equation}
\iiint_{\Omega_{cv}} \delta \overline{V} \, dV =  (\delta \overline{V})_P \, \Delta x \Delta y
\end{equation}

Resolviendo $(***)$

\begin{align*}
\begin{split}
\dfrac{2 \Delta t}{3 \, Re} \iiint_{\Omega_{cv}}  \dfrac{\partial^2 (\delta \overline{V}) }{\partial x^2} \, dV &= \dfrac{2 \Delta t}{3 \, Re} \iint_{\partial \Omega_{cv}} \dfrac{\partial (\delta \overline{V})}{\partial x} \cdot \vec{n} \, dA \\
&= \dfrac{2 \Delta t}{3 \, Re} \left[ -\left(\dfrac{\delta \overline{V}_E - \delta \overline{V}_P}{\Delta x} \right) + \left( \dfrac{\delta \overline{V}_P-\delta \overline{V}_W}{\Delta x} \right) \right] \, \Delta x \Delta y \\
&= \dfrac{2 \Delta t \Delta y}{3 \, Re} \left[ -\delta \overline{V}_E + 2 \delta \overline{V}_P - \delta \overline{V}_W \right]
\end{split}
\end{align*}

Reemplazando $(**)$ y $(***)$ en la (\ref{primer_paso}) se obtiene:

\begin{equation}
(\delta \overline{V})_P \, \Delta x \Delta y - \dfrac{2 \Delta t \Delta y}{3 \, Re} \left[ -\delta \overline{V}_E + 2 \delta \overline{V}_P - \delta \overline{V}_W \right] = \iiint_{\Omega_{cv}} RHS^n \, dV
\end{equation}

El término $RHS^n$ se reemplaza por la ecuación la discretización planteada en la Sección \ref{convectivo} y por el campo de presión definido arbitrariamente a \textit{priori}. La ecuación anterior se traduce en resolver un sistemas de ecuaciones tridiagonal dado por,

\begin{equation}
\begin{split}
\left( \dfrac{2 \Delta t \Delta y}{3 \, Re} \right) \, \delta \overline{V}_E + \left( \Delta x \Delta y - \dfrac{4 \Delta t \Delta y}{3 \, Re} \right) \delta \overline{V}_P + \left( \dfrac{2 \Delta t \Delta x}{3 \, Re} \right) \, \delta \overline{V}_W \\
= \iiint_{\Omega_{cv}} RHS^n \, dV
\end{split}
\end{equation}

\subsubsection{2do paso en la dirección $y$ ($\delta V$)}

Se procede al igual que en el primer paso: 

\begin{equation} \label{segundo_paso}
\iiint_{\Omega} \Psi_i \left[ \left( I - \dfrac{2 \Delta t}{3 \, Re}  \dfrac{\partial^2}{\partial y^2} \right) \delta V_i - \delta \overline{V}_i \right] \, dV = 0
\end{equation}

\begin{equation}
\iiint_{\Omega_{cv}} \left( I - \dfrac{2 \Delta t}{3 \, Re}  \dfrac{\partial^2}{\partial y^2} \right) \delta V \, dV =  \underbrace{\iiint_{\Omega_{cv}} \delta V \, dV}_{('')}  - \underbrace{\dfrac{2 \Delta t}{3 \, Re} \iiint_{\Omega_{cv}}  \dfrac{\partial^2 (\delta \overline{V}) }{\partial y^2} \, dV}_{(''')}
\end{equation}

Resolviendo $('')$
\begin{equation}
\iiint_{\Omega_{cv}} \delta V \, dV =  (\delta V)_P \, \Delta x \Delta y
\end{equation}

Resolviendo $(''')$
\begin{align*}
\begin{split}
\dfrac{2 \Delta t}{3 \, Re} \iiint_{\Omega_{cv}}  \dfrac{\partial^2 (\delta V) }{\partial x^2} \, dV &= \dfrac{2 \Delta t}{3 \, Re} \iint_{\partial \Omega_{cv}} \dfrac{\partial (\delta V)}{\partial y} \cdot \vec{n} \, dA \\
&= \dfrac{2 \Delta t}{3 \, Re} \left[ -\left(\dfrac{\delta V_N - \delta V_P}{\Delta y} \right) + \left( \dfrac{\delta V_P-\delta V_S}{\Delta y} \right) \right] \, \Delta x \Delta y \\
&= \dfrac{2 \Delta t \Delta x}{3 \, Re} \left[ -\delta V_N + 2 \delta V_P - \delta V_S \right]
\end{split}
\end{align*}

Reemplazando $('')$ y $(''')$ en la (\ref{segundo_paso}) se obtiene:

\begin{equation}
(\delta V)_P \, \Delta x \Delta y - \dfrac{2 \Gamma \Delta t \Delta x}{3 \, Re} \left[ -\delta V_N + 2 \delta V_P - \delta V_S \right] = \iiint_{\Omega_{cv}} (\delta \overline{V} ) \, dV
\end{equation}

El término del lado derecho de la ecuación se obtuvo del primer paso de tiempo. Luego, se resuelve el sistema tridiagonal dado por,

\begin{equation}
\begin{split}
\left( \dfrac{2 \Gamma \Delta t \Delta x}{3 \, Re} \right) \, \delta V_N + \left( \Delta x \Delta y - \dfrac{4 \Gamma \Delta t \Delta x}{3 \, Re} \right) \delta V_P + \left( \dfrac{2 \Gamma \Delta t \Delta x}{3 \, Re} \right) \, \delta V_S \\
= \iiint_{\Omega_{cv}} (\delta \overline{V} ) \, dV
\end{split}
\end{equation}

\paragraph{Observación} los pasos anteriores se realizan tanto para $u$ como para $v$ ya que $\vec{v}=\left\{ u,v \right\}^T$ y $\delta V = \vec{v}^* - \vec{v}^n $. Cada paso de tiempo implica resolver 4 sistemas tridiagonales: 2 sistemas para la componente horizontal y 2 sistemas para la componente vertical de la velocidad. Finalmente se obtiene el predictor de la velocidad:

\begin{equation}
\vec{v}^* = \Big| \begin{matrix} u^* = u^n + \delta V_x \\ v^* = v^n + \delta V_y
\end{matrix}
\end{equation}

\subsection{Resolución de la ecuación de Poisson sobre la presión}

\begin{equation}
\Delta \phi = \dfrac{3}{2 \Delta t} \nabla \cdot \vec{v}^*
\end{equation}

\begin{equation}
\phi = \left( P^{n+1} - P^n \right) + \dfrac{1}{Re} \nabla \cdot \vec{v}^* 
\end{equation}

Nuevamente se aplica el método de residuos ponderados con formulación para volúmenes finitos.

\begin{equation}
\iiint_{\Omega} \Psi_i  \left[ \Delta \phi - \dfrac{3}{2 \Delta t} \nabla \cdot v_i^* \right] \, dV = 0
\end{equation}

Luego,

\begin{equation}
\iiint_{\Omega_J} \Delta \phi \, dV = \iiint_{\Omega_J} \dfrac{3}{2 \Delta t} \nabla \cdot v_i^* \, dV
\end{equation}

Recurriendo al teorema de Green-Ostrogradsky se rescribe la ecuación:

\begin{equation}
\iint_{\partial \Omega_J} \vec{\nabla} \phi \vec{n} \, dA = \dfrac{3}{2 \Delta t} \iint_{\partial \Omega_J} \vec{v}^* \cdot \vec{n} \, dA
\end{equation}

\paragraph{Discretización} La función auxiliar $\phi$ se discretiza en la malla desplazada asociada a la presión. Luego, los valores de los flujos de $\phi$ se aproximan en los bordes de cada volumen finito $\Omega_J$, mientras que los valores de las velocidad en $\partial \Omega_J$ son conocidos, sin necesidad de aproximarlos.

\begin{equation}
\begin{split}
\left[ \dfrac{\partial \phi}{\partial x} \Big|_e - \dfrac{\partial \phi}{\partial x} \Big|_w \right] \Delta y + \left[ \dfrac{\partial \phi}{\partial y} \Big|_n - \dfrac{\partial \phi}{\partial y} \Big|_s \right] \Delta x = \\ \dfrac{3}{2 \Delta t} \left( u^*_e - u_w^*  \right) + \dfrac{3}{2 \Delta t} \left( v^*_n - v_s^*  \right)
\end{split}
\end{equation}

Se discretizan las derivadas de $\phi$ implementando un esquemas CDS, obteniendose:

\begin{equation} \label{poisson_discreto}
\begin{split}
\left[ \dfrac{\phi_E-2\phi_P+\phi_W}{\Delta x} \right] \Delta y - \left[ \dfrac{\phi_N-2\phi_P+\phi_S}{\Delta y} \right] \Delta x = \\ \dfrac{3}{2 \Delta t} \left( u_e^* - u_w^* + v_n^* - v_s^* \right)
\end{split}  
\end{equation}

Agrupando los términos $\phi_{nb}$ resulta un sistema de ecuaciones lineales que en forma matricial se representa por una matriz pentadiagonal. Para resolverlo se utiliza el algoritmo TDMA (\textit{Tri-Diagonal MAtrix Algorithm}). Para ellos se define el término $B$ como,

\begin{equation}
B =  \dfrac{3}{2 \Delta t} \left( u_e^* - u_w^* + v_n^* - v_s^* \right)
\end{equation}

Se calcula un predictor (I):

\begin{equation}
\begin{split}
\left( \dfrac{\Delta y}{\Delta x} \right) \phi_E^I - \left( 2 \dfrac{\Delta y}{\Delta x} + 2 \dfrac{\Delta x}{\Delta y} \right) \phi_P^I + \left( \dfrac{\Delta y}{\Delta x} \right) \phi_W^I = \\ \left( \dfrac{\Delta x}{\Delta y} \right) \phi_N + \left( \dfrac{\Delta x}{\Delta y} \right) \phi_S + B
\end{split}
\end{equation}

Se calcula una corrección (II):

\begin{equation}
\begin{split}
\left( \dfrac{\Delta x}{\Delta y} \right) \phi_N^{II} - \left( 2 \dfrac{\Delta y}{\Delta x} + 2 \dfrac{\Delta x}{\Delta y} \right) \phi_P^{II} + \left( \dfrac{\Delta x}{\Delta y} \right) \phi_S^{II} = \\ \left( \dfrac{\Delta y}{\Delta x} \right) \phi_E^I + \left( \dfrac{\Delta y}{\Delta x} \right) \phi_W^I + B
\end{split}
\end{equation}

Un criterio de detención que suele utilizarse en establece lo siguiente: En cada punto de la malla se calcula un residuo $R$ como,

\begin{equation}
R = \sum a_{nb} \phi_{nb} + B - a_P \phi_P 
\end{equation}

La solución converge en la medida que $R$ tiende a cero. Más detalles del algoritmo se explican en \cite{patankar} y \cite{versteeg}. 

\subsection{Correción del campo de velocidad}

Una vez resuelto la ecuación de Poisson se procede a corregir la velocidad predicha en el primer paso:

\begin{equation}
\vec{v}^{n+1} = \vec{v} - \Delta t \nabla \phi
\end{equation}

Se recurre a la misma discretización utilizada en la ecuación (\ref{poisson_discreto}) para el campo auxiliar $\phi$.

